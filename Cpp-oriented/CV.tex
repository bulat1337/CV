
\documentclass[letterpaper,11pt]{article}

\usepackage{latexsym}
\usepackage[empty]{fullpage}
\usepackage{titlesec}
\usepackage{marvosym}
\usepackage[usenames,dvipsnames]{color}
\usepackage{verbatim}
\usepackage{enumitem}
\usepackage[hidelinks]{hyperref}
\usepackage{fancyhdr}
\usepackage[english, russian]{babel}
\usepackage{tabularx}
\usepackage{indentfirst}
\setlength{\parindent}{1cm}

\input{glyphtounicode}


%----------FONT OPTIONS----------
% sans-serif
% \usepackage[sfdefault]{FiraSans}
% \usepackage[sfdefault]{roboto}
% \usepackage[sfdefault]{noto-sans}
% \usepackage[default]{sourcesanspro}

% serif
% \usepackage{CormorantGaramond}
% \usepackage{charter}


\pagestyle{fancy}
\fancyhf{} % clear all header and footer fields
\fancyfoot{}
\renewcommand{\headrulewidth}{0pt}
\renewcommand{\footrulewidth}{0pt}

% Adjust margins
\addtolength{\oddsidemargin}{-0.5in}
\addtolength{\evensidemargin}{-0.5in}
\addtolength{\textwidth}{1in}
\addtolength{\topmargin}{-.5in}
\addtolength{\textheight}{1.0in}

\urlstyle{same}

\raggedbottom
\raggedright
\setlength{\tabcolsep}{0in}

% Sections formatting
\titleformat{\section}{
  \vspace{-4pt}\scshape\raggedright\large
}{}{0em}{}[\color{black}\titlerule \vspace{-5pt}]

% Ensure that generate pdf is machine readable/ATS parsable
\pdfgentounicode=1

%-------------------------
% Custom commands
\newcommand{\resumeItem}[1]{
  \item\small{
    {#1 \vspace{-2pt}}
  }
}

\newcommand{\resumeSubheading}[4]{
  \vspace{-2pt}\item
    \begin{tabular*}{0.97\textwidth}[t]{l@{\extracolsep{\fill}}r}
      \textbf{#1} & #2 \\
      \textit{\small#3} & \textit{\small #4} \\
    \end{tabular*}\vspace{-7pt}
}

\newcommand{\resumeSubSubheading}[2]{
    \item
    \begin{tabular*}{0.97\textwidth}{l@{\extracolsep{\fill}}r}
      \textit{\small#1} & \textit{\small #2} \\
    \end{tabular*}\vspace{-7pt}
}

\newcommand{\resumeProjectHeading}[2]{
    \item
    \begin{tabular*}{0.97\textwidth}{l@{\extracolsep{\fill}}r}
      \small#1 & #2 \\
    \end{tabular*}\vspace{-7pt}
}

\newcommand{\resumeSubItem}[1]{\resumeItem{#1}\vspace{-4pt}}

\renewcommand\labelitemii{$\vcenter{\hbox{\tiny$\bullet$}}$}

\newcommand{\resumeSubHeadingListStart}{\begin{itemize}[leftmargin=0.15in, label={}]}
\newcommand{\resumeSubHeadingListEnd}{\end{itemize}}
\newcommand{\resumeItemListStart}{\begin{itemize}}
\newcommand{\resumeItemListEnd}{\end{itemize}\vspace{-5pt}}

%-------------------------------------------
%%%%%%  RESUME STARTS HERE  %%%%%%%%%%%%%%%%%%%%%%%%%%%%


\begin{document}

%----------HEADING----------
% \begin{tabular*}{\textwidth}{l@{\extracolsep{\fill}}r}
%   \textbf{\href{http://sourabhbajaj.com/}{\Large Sourabh Bajaj}} & Email : \href{mailto:sourabh@sourabhbajaj.com}{sourabh@sourabhbajaj.com}\\
%   \href{http://sourabhbajaj.com/}{http://www.sourabhbajaj.com} & Mobile : +1-123-456-7890 \\
% \end{tabular*}

\begin{center}
    \textbf{\Huge \scshape Мотыгуллин Булат} \\ \vspace{1pt}
    \small +79172841606 $|$ \href{bulatmot@gmail.com}{\underline{bulatmot@gmail.com}} $|$ 
    \href{https://t.me/bulatttttttllllll}{\underline{t.me/bulatttttttllllll}} $|$
    \href{https://github.com/bulat1337}{\underline{github.com/bulat1337}}
\end{center}


%-----------EDUCATION-----------
\section{Образование}
\resumeSubHeadingListStart
  \resumeSubheading
    {МФТИ - Московский физико-технический университет}{Москва}
    {Бакалавр ФРКТ - Физтех-школа радиотехники и комьютерных технологий}{2023 -- 2027 $|$ 2 курс}
\resumeSubHeadingListEnd

\vspace*{-3mm} % Уменьшение вертикального отступа между блоками

\resumeSubHeadingListStart
  \resumeSubheading
    {Основные курсы:}{}
    {Математический Анализ, Линейная Алгебра, Теория Вероятности, Информатика, Физика}{}
\resumeSubHeadingListEnd

%-----------PROJECTS-----------
\section{Проекты}
    \resumeSubHeadingListStart
      \resumeProjectHeading
          {\textbf{Tatlang} $|$ \emph{C, Make, Git, CPU emulator, Binary tree, Recursive descent parser}}
          {\href{https://github.com/bulat1337/Tatlang}{\underline{github.com/bulat1337/Tatlang}}}
          \resumeItemListStart
            \resumeItem{Написал язык программирования. Синтаксис адаптирован под татарский язык и поддерживает все символы Unicode таблицы.}
            \resumeItem{Разработал компилятор. Основные этапы работы: токенизация, формирование абстрактного синтаксического дерева (АСД) и ассемблирование.}
            \resumeItem{Адаптировал процесс ассемблирования под эмулятор процессора.}
          \resumeItemListEnd
    \resumeProjectHeading
          {\textbf{Ray Tracing} $|$ \emph{C++, CMake, Git}}
          {\href{https://github.com/bulat1337/Ray_Tracing}{\underline{github.com/bulat1337/Ray\_Tracing}}}
          \resumeItemListStart
            \resumeItem{Разработал движок трассировки лучей, который вручную рассчитывает путь каждого луча в 3D-сцене.}
            \resumeItem{Реализовал механизмы сглаживания, глубины диффузии и эффекта размытия, что позволяет пользователям настраивать баланс между качеством изображения и производительностью.}
            \resumeItem{Оптимизировал процесс рендеринга с помощью структуры BVH (Bounding Volume Hierarchy) для ускорения проверки пересечений лучей с объектами.}
          \resumeItemListEnd
    \resumeProjectHeading
          {\textbf{CPU} $|$ \emph{C, Make, Git, Stack, SFML}}
          {\href{https://github.com/bulat1337/CPU}{\underline{github.com/bulat1337/CPU}}}
          \resumeItemListStart
            \resumeItem{Эмулировал работу процессора. Основные этапы работы процессора: формирование байт-кода и его исполнение.}
            
            \resumeItem{Разработал язык ассемблера.}
            
            \resumeItem{Добавил поддержку арифметических операций, ввода и вывод данных. Реализовал работу с регистрами, стеком и оперативной памятью.}
            
            \resumeItem{Эмулировал видеопамять. Для формирования изображения использовал графическую библиотеку SFML.} 
          \resumeItemListEnd
    \resumeProjectHeading
          {\textbf{Differentiator} $|$ \emph{C, Binary tree, Recursive descent parser, Graphviz}}
          {\href{https://github.com/bulat1337/Differentiator}{\underline{github.com/bulat1337/Differentiator}}}
          \resumeItemListStart
            \resumeItem{Разработал программу для дифференцирования и упрощения математических выражений.}
            \resumeItem{Добавил генерацию описания процесса работы в формате Tex.}
            \resumeItem{Выражения представлены в формате бинарного дерева. Для графического представления деревьев использовал Graphviz.}
            \resumeItem{Реализовал анализ математических выражений с помощью алгоритма рекурсивного спуска.}
          \resumeItemListEnd
    \resumeProjectHeading
          {\textbf{Akinator} $|$ \emph{C, Make, Git, Binary Tree, Graphviz}}
          {\href{https://github.com/bulat1337/Akinator}{\underline{github.com/bulat1337/Akinator}}}
          \resumeItemListStart
            \resumeItem{Написал игру, которая угадывает загаданный объект, задавая пользователю вопросы}
          \resumeItemListEnd
    \resumeProjectHeading
          {\textbf{Range Queries} $|$ \emph{C++, CMake, Git, Red–black Tree, Graphviz}}
          {\href{https://github.com/bulat1337/Range_Queries}{\underline{https://github.com/bulat1337/Range\_Queries}}}
          \resumeItemListStart
            \resumeItem{Создал интерактивную программу на основе своего черно-красного дерева, позволяющую добавлять узлы и запрашивать их количество в заданных границах.}
          \resumeItemListEnd
    \resumeProjectHeading
          {\textbf{Triangle Intersection} $|$ \emph{C++, CMake, Git, Spacial Hashing}}
          {\href{https://github.com/bulat1337/Triangle_Intersection}{\underline{https://github.com/bulat1337/Triangle\_Intersection}}}
          \resumeItemListStart
            \resumeItem{Реализовал алгоритм, проверяющий пересечаение треугольников в трехмерном пространстве.}
            \resumeItem{Оптимизировал алгоритм для обработки пересечений большого количества треугольников, применив пространственное хеширование (Spatial Hashing).}
          \resumeItemListEnd
    \resumeProjectHeading
          {\textbf{Adaptive Replacement Cache} $|$ \emph{C++, CMake, Git}}
          {\href{https://github.com/bulat1337/ARC}{\underline{https://github.com/bulat1337/ARC}}}
          \resumeItemListStart
            \resumeItem{Реализовал адаптивное кеширование данных}
            \resumeItem{Написал "идеальный" алгоритм кеширования, для сравнения реазультатов.}
          \resumeItemListEnd

    \resumeSubHeadingListEnd



%
%-----------О СЕБЕ-----------
\section{О себе}

\hspace{1cm}Учась в школе, я участвовал в олимпиадах по физике, стал призёром в:
\vspace*{-1.5mm}
    \resumeItemListStart
        \resumeItem{Инженерная олимпиады школьников (физика) - призёр}
\vspace*{-2.5mm}
        \resumeItem{Олимпиада Курчатов физика - призёр}
    \resumeItemListEnd
    
\newline

\vspace*{+1mm}

\hspace{1cm}На физтехе я получил навык работы в команде, научился грамотному техническому общению. Опыт мозгового штурма и совместного решения сложных задач поможет мне работать в профессиональной команде, прислушиваться к каждому и
эффективно находить лучшее решение проблемы. 



%-------------------------------------------
\end{document}
